\documentclass[french]{article}
\usepackage[utf8]{inputenc}
\usepackage[T1]{fontenc}
\usepackage{babel}
\usepackage{amssymb}
\usepackage{amsmath}
\usepackage{titlesec}
\usepackage{graphicx}
\usepackage{xr}
\usepackage{biblatex}
\usepackage{pdfpages}



\date{September 23, 2021}
\author{\bsc{Réda Bensahla - Université de Lille} }

\title{Le Mouvement Brownien Fractionnaire appliqué à la Log-Volatilité des actifs financiers (Rough Volatility)}

\begin{document}
   \maketitle 
   \newpage
   \renewcommand{\contentsname}{Sommaire}
      \tableofcontents

   \newpage   
   \section{Introduction}

      \subsection{L'utilisation du mouvement brownien dans les modèles financiers modernes}
	Dans la modélisation financière moderne, les prix ont toujours été modélisés par des martingales semi-continues, de plus, il a toujours été nécessaire de présenter certaines hypothèses, qui aujourd'hui peuvent être remises en cause, notamment :
	\begin{itemize}
    		\item L'hypothèse de normalité des rendements, ceux-ci présentent en général des asymétries et des queues épaisses.
   		\item La continuité des cours, pour cela il suffit d'observer les sauts, plus communément appelés "gaps" entre deux instants de quotations de cours.
    		\item Mais surtout, et le plus important, l'hypothèse d'indépendance des accroissements, en effet, il s'agit ici d'ignorer l'impacts des évènements réalisés dans le passé, pour cela, Mandelbrot préconise d'utiliser le mouvement Brownien Fractionnaire à la place du mouvement Brownien.
	\end{itemize} 
	C'est donc ce point qui nous a amené à faire cette étude sur le mouvement Brownien Fractionnaire, de plus, la Log-Volatilité étant fractionnaire l'utilisation de ce mouvement Brownien nous donne donc des modèles assez consistant, cependant, comme nous le verrons par la suite, il est assez difficile à mettre en place.
     
      \subsection{Modélisation de la volatilité}
      Les Logs-prix sont souvent modélisés comme des semi-martingales continues. Pour un actif donné avec un Log-Prix $Y_t$, ce dernier peut être modélisé comme suit : \[ dY_t = \mu_t dt + \sigma_t dWt\] où $\mu_t$ est le terme de drift et $W_t$ est un mouvement brownien uni-dimensionnel.
      Le terme $\sigma_t$ représente quant à lui le processus de volatilité du modèle.
      D'un côté nous avons des modèles comme le modèle de Black-Scholes où la volatilité est souvent constante ou est une fonction déterministe du temps, puis d'un autre côté, nous avons des modèles à volatilité stochastiques, la volatilité $\sigma_t$ est modélisée par une semi-martingale brownienne, parmi ces modèles nous pouvons citer le                                 	modèle de Heston dont voici la dynamique de la volatilité: \[ d\nu_t = \kappa(\theta - \nu_t)dt + \xi \sqrt{\nu_t} dB_t\] Ou encore le modèle CEV ("Constant Elasticity of Variance Model") dont la volatilité est stochastique mais déterministe, sa dynamique s'écrit comme suit :  \[ dS_t = \mu S_tdt + \sigma S_t^\gamma dB_t\]
      
      \subsection{Le mouvement brownien fractionnaire}
      	\subsubsection{Définition du mouvement brownien fractionnaire}
     	  Dans cette section, nous allons définir le mouvement brownien fractionnaire, ses propriétés, comment le simuler et enfin comment estimer l'index de Hurst (H).\newline \newline Le mouvement brownien fractionnaire noté $\{B_H(t)\}_{t\in\mathbb{R}}$ d'exposant de Hurst $H\in ]0;1]$, est l'unique processus gaussien centré, nul en zéro, dont les 	  accroissements sont stationnaires et auto-similaires, il est défini par  sa fonction de covariance $\Gamma(t,s)$ : \[  \Gamma(t,s) = \mathbb{E} [B^H(s)B^H(t)] = \frac{C(H)}{2}(\lvert t \rvert^{2H}  +   \lvert s \rvert^{2H}      +     \lvert t - s \rvert^{2H}))  \]
	  Sa fonction d'auto-covariance $\gamma(t,s)$ est donnée par : 
	  \[  \gamma(t,s) =  \frac{C(H)}{2}(\lvert t-s-1 \rvert^{2H}  -   2\lvert t-s \rvert^{2H}      +     \lvert t-s+1 \rvert^{2H}))  \]
	  où \[ C(H) = Var[B^H(1)]\] 
      	  Où : $B^H(0) = 0$ et $B(s)$ représente le mouvement brownien standard \newline
    	  Lorsque $C(H) = 1$ ce processus est appelé Mouvement Brownien Fractionnaire standard, de plus lorsque H = $\frac{1}{2}$, le mouvement brownien fractionnaire correspond à un mouvement brownien standard \newline
	  
	\subsubsection{Représentation de Mandelbrot-Van-Ness}
     	  Une autre forme de représentation du mouvement brownien fractionnaire est celle de Mandelbrot-Van-Ness, aussi appelé moving-average representation, celle-ci est donnée par :
	  \[ B^H(t) := \frac{1}{C(H)}\int_{\mathbb{R}} [ (t-s)_+^{H-\frac{1}{2}}-(-s)_+^{H-\frac{1}{2}}\ ]\mathrm{d}B(s),\] 
	  Avec : $C(H) = \left(  \int_{0}^{\infty} \left((1+s)^{H-\frac{1}{2}} - s^{H - \frac{1}{2}} \right)^2ds + \frac{1}{2H}      \right)^{\frac{1}{2}}$
	  
	  Nous allons démontrer ce théorème : \newline
	  On définit $f_t(s) = ((t-s)^+)^{H-\frac{1}{2}} - ((-s)^+)^{H-\frac{1}{2}}, \;\;\; s \in \mathbb{R}, t \leq 0$ \newline
	  Il est nécessaire de souligner que dans la définition de l'intégrale de Wiener l'opérateur intégrant doit être dans $L^2$, autrement dit : $\int_{\mathbb{R}} f_t(s)^2ds < \infty$, \newline Lorsque $H \neq \frac{1}{2}$, $s \rightarrow -\infty$, $f_t(s)$ tends vers $(-s)^{H-\frac{3}{2}} $, de carré intégrable, et pour $t \geq 0$ posons : 
	  \[ X_t = \int_{\mathbb{R}} \left[ ((t-s)+)^{H-\frac{1}{2}} - ((-s)^+)^{H-\frac{1}{2}}   \right]dB_s \]
	  On a : 
	  \[ \mathbb{E}[X_t^2] = \int_{\mathbb{R}} \left[ ((t-s)+)^{H-\frac{1}{2}} - ((-s)^+)^{H-\frac{1}{2}}   \right]^2ds \]
	  \[ = t^{2H}\int_{\mathbb{R}} \left[ ((1-u)^+)^{H-\frac{1}{2}} - ((-u)^+)^{H-\frac{1}{2}}              \right]^2du\]
	  \[ = t^{2H}     \left(\int_{-\infty}^{0} [(1-u)^{H-\frac{1}{2}} - (-u)^{H-\frac{1}{2}}]^2du + \int_{0}^{1}(1-u)^{2H-1}du             \right)  \]
	  \[ = C(H)^2t^{2H}\]
	  On a aussi, pour : 
	  \[ \mathbb{E}[\vert X_t - X_s \rvert] = \int_{R} \left[  ((t-u)^+)^{H - \frac{1}{2}} - ((s-u)^+)^{H - \frac{1}{2}}    \right]^2du \]
	  \[ \mathbb{E}[\vert X_t - X_s \rvert] = \int_{R} \left[  ((t-s-u)^+)^{H - \frac{1}{2}} - ((-u)^+)^{H - \frac{1}{2}}    \right]^2du \]
	  \[ \mathbb{E}[\vert X_t - X_s \rvert] = C(H)^2\lvert t-s \rvert^{2H} \] \newline
	  
	  Malheureusement, comme nous allons le montrer par la suite, le mouvement brownien fractionnaire n'est pas une semi-martingale pour $H \neq \frac{1}{2}$, on ne peut donc pas lui appliquer toute la théorie d'Itô sur l'intégration stochastique des semi-martingales \newline
	  Démontrons cette précédente affirmation : 
	  Posons $p > 0$ : 
	  
	  \[ Y_{n, p} = n^{pH-1}\sum_{j=1}^n\lvert B_{j/n} - B_{(j-1)/n} \rvert  ^p \]
	  Par la propriété d'auto-similarité du mouvement brownien fractionnaire, la suite $(Y_{n,p}, \;\; n \geq 1)$ a la même distribution que $(\tilde{Y}_{n,p}, n \geq 1)$, avec : 
	  \[  \tilde{Y}_{n, p} = n^{-1}\sum_{j=1}^n \lvert B_j - B_{j-1} \rvert  ^p \]
	  Par le théorème ergodique on sait que $\tilde{Y}_{n, p}$ converge presque sûrement vers $\mathbb{E}[\lvert B_1 \rvert ^p ]$ quand n tend vers l'infini, donc $Y_{n,p}$ converge en probabilité quand n tend vers l'infini vers $\mathbb{E}[\lvert B_1 \rvert ^p ]$, et on a donc : 
	  \[  V_{n,p} = \sum_{j=1}^n\lvert B_{j/n} - B_{(j-1)/n} \rvert  ^p  \]
	  Converge en probabilité vers 0 quand n tend vers l'infini si $pH < 1$, il est nécessaire de distinguer deux cas :
	  \begin{itemize}
		\item Si $H < \frac{1}{2}$, prenons $p > 2$ tel que $pH < 1$ ce qui a pour conséquence que la p-variation du mouvement brownien fractionnaire est infini, pour $p=2$ la variation quadratique est aussi finie.
		\item Si $H > \frac{1}{2}$, on peut choisir p tel que $\frac{1}{H} < p < 2$, alors la p-variation vaut 0, et, en conséquence, la variation quadratique est aussi 0. \newline De plus, si on choisit p tel que $1 < p < \frac{1}{H}$, on déduit que la variation quadratique totale est infinie.
	 \end{itemize}
	 Ainsi, nous avons démontré que pour $H \ne \frac{1}{2}$ le mouvement brownien fractionnaire ne peut donc pas être une semi-martingale. \newline
	 
	  L'idée étant de comprendre les p-variations du mbf et de montrer qu'e
	  
	  Le comportement du mouvement brownien fractionnaire dépend donc de ce paramètre H : 
	  \begin{itemize}
    		\item Lorsque $H < \frac{1}{2}$ les trajectoires sont assez irrégulières (comparativement au mouvement brownien standard).
   		\item Lorsque $H > \frac{1}{2}$ les trajectoires sont plus régulières, on constate l'effet de longue mémoire.
	\end{itemize} 
	  
	  \subsubsection{Simulation du mouvement brownien fractionnaire} 
	  Nous allons donc présenter ici une méthode de simulation du mouvement brownien fractionnaire : La méthode de Cholesky \newline
	  	\paragraph{Méthode de Cholesky}
		Il s'agit ici d'une méthode qui est exacte en théorie, mais qui n'est malheureusement pas très utile en pratique car son execution nécessite beaucoup de temps. \newline
		Posons $\Gamma$ la matrice de covariance du mouvement brownien discrétisé aux instants $\frac{i}{n}$, pour $i = 0,\ldots,n-1.$ \newline
		La méthode de Cholesky consiste à déduire $\Gamma^\prime$ de $\Gamma$ en supprimant la première ligne et la première colonne, la matrice $\Gamma^\prime$ est symétrique définie positive et admet donc
		une décomposition de Cholesky $\Gamma^\prime = LL^t$ où $L$ est est une matrice triangulaire inférieure.\newline \newline
		Par la suite, on effectue le produit matriciel $LZ$ où $Z$ est un vecteur de $n-1$ variables aléatoires indépendantes gaussiennes centrées et réduite, le vecteur $LZ$ est un vecteur gaussien centré et on a : $\mathbb{E}[(LZ)(LZ)^t] = \Gamma^\prime$ \newline \newline
	  	Ainsi, le vecteur $B_H = (0, (LZ)^t)^t$ représente une trajectoire du mouvement brownien discrétisé.
			
	   
      \subsubsection{Quelques simulations du mouvement brownien fractionnaire \dots }
      
     	 \begin{figure}[!h]
  		\includegraphics[width=\linewidth]{SimulationsMouvementBrownienFractionnaireH.png}
  		\caption{Simulation du mouvement brownien fractionnaire pour différentes valeurs de H}
 		\label{fig:Simulation du mouvement brownien fractionnaire pour différentes valeurs de H}
	\end{figure}
	\newpage
	
      \subsection{La volatilité rugueuse}
      	\subsubsection{Définition}
      		Les différents modèles historiques dans la modélisation financière moderne proposent deux approches concernant le processus de volatilité $\sigma_t$ : 
		\begin{itemize}
    			\item Premièrement, des trajectoires assez régulières (constantes ou déterministes), dans le cas du modèle de Black-Scholes standard.
   			\item Deuxièmement, dans le cas des modèles à volatilité locales ou stochastiques, comme le modèle de Heston, les trajectoires de volatilité modélisées correspondaient pratiquement à celles du mouvement brownien standard.
		\end{itemize}
	Dans l'article de Comte et Renault, les auteurs proposent, en partant du fait que la volatilité est un processus à mémoire-longue, de modéliser la log-volatilité en utilisant un mouvement brownien fractionnaire avec pour paramètre $H \in ]\frac{1}{2},1[$, ce modèle est appelé Fractional Stochastic Volatility (FSV) \newline \newline
	
	Dans l'article de Gatheral, Jaisson et Rosenbaum, un modèle plus complexe est proposé, le Rough Fractional Stochastic Volatility (RFSV), celui-ci est le même que le modèle FSV mais en prenant $H \in ]0; \frac{1}{2}[$, nous détaillerons ces modèles dans la section ().\newline \newline
	
	Nous utiliserons par la suite le mouvement brownien fractionnaire afin de modéliser les incréments de la log-volatilité, en partant du principe que les incréments stationnaires du mouvement brownien fractionnaire satisfont, pour tout $t \in \mathbb{R}, \Delta \geq 0, q>0$ : 
	 \[ \mathbb{E}[\lvert B_{t+\Delta}^H -  B_t^H \rvert^q]  = K_q\Delta^{qH}     \] 
	 Lorsque $H > \frac{1}{2}$, les incréments du mouvement brownien fractionnaire sont positivement corrélés et montre un effet de mémoire longue, en effet, la trajectoire de $B_{t+1}^H$ dépend de $B_t^H$, et on a donc :  	 \[  \sum_{k=0}^{+\infty}Cov[W_1^H, W_k^H - W_{k-1}^H] = +\infty   \]  \newline
	 En effet, $Cov[W_1^H, W_k^H - W_{k-1}^H]$ est de l'ordre $k^{2H-2}$ avec $k\longrightarrow \infty$
	 
      	Où $K_q$ représente le moment d'ordre $q$ de la valeur absolue d'une variable gaussienne centrée et réduite; $\Delta$ représente la variation temporelle. \newline \newline
	L'approche de Gatheral et Rosenbaum est de partir du principe que l'on a accès à différentes observations du processus de volatilité $\sigma_t$ et de découper les instants de volatilités sur $[0;T]$ comme suit : $\sigma_0, \ldots, \sigma_\Delta, \dots, \sigma_{k\Delta}, \dots: k \in {[0,  N]}$, où $N = \left\lfloor {T/\Delta} \right\rfloor$.
	
	On définit donc, pour $q \geq 0$ :
	\[ m(q, \Delta) = \frac{1}{N} \sum_{k=1}^N\lvert log(\sigma_{k\Delta}) - log(\sigma_{(k-1)\Delta}) \rvert^{q}  \] 
	
	\subsubsection{Estimation de l'index de Hurst H}
	La méthode d'estimation du paramètre H proposée par Gatheral, Jaisson et Rosenbaum se construit comme suit : \newline \newline
	Il faut comprendre le comportement de la quantité $\frac{log \; m(q,\Delta)}{log(\Delta)}$, on s'aperçoit que cette relation est linéaire, à un coefficient $\zeta_q$ près, on s'aperçoit (cf Figure 2) que pour tout $q : \; m(q, \Delta) \propto \Delta^{\zeta_q} $ \newline
	\begin{figure}[!h]
  		\includegraphics[width=\linewidth, height=80mm]{logM(q,delta)logdelta.png}
  		\caption{Relation entre $log(m(q, \Delta))$ et $log(\Delta)$; Calculé sur l'indice CAC 40}
 		\label{fig:}
	\end{figure}\newpage
	Enfin il s'agit d'étudier le comportement de $\zeta_q$ en fonction de $q$, nos résultats nous montrent qu'il existe donc une relation linéaire entre $\zeta_q$ et $q$, de ce résultat est déduit la pente la droite qui correspond donc au paramètre H (cf Figure 3), et on déduit donc la relation d'échelle monofractal suivante :  \[  \zeta_q = qH \]
	Il convient de remarquer que H varie très peu en fonction du temps.
	\begin{figure}[!h]
  		\includegraphics[width=\linewidth, height=80mm]{RelationH.png}
  		\caption{Relation entre $\zeta_q$ et $q$}
 		\label{fig:}
	\end{figure}\newpage

	\subsubsection{Application et estimation du paramètre de Hurst H sur d'autres indices}
	On applique la même méthode sur différentes volatilités de différents indices, les résultats sont les suivants : 
	\begin{figure}[!h]
  		\includegraphics[width=\linewidth, height=150mm]{EstimationHetNuIndices.png}
  		\caption{Estimation du paramètre H de différents indices}
 		\label{fig:}
	\end{figure}
	
   \subsection{Équations Différentielles Stochastiques dirigées par un mBF}
   Dans cette section nous allons nous intéresser à la résolution théorique de certaines EDS ou lorsque la solution théorique n'est pas atteignable, nous présenterons un schéma de résolution numérique (Schéma de Milstein).
   Enfin, nous parlerons des équations dirigés par le mouvement brownien fractionnaire tels que le processus d'Ornstein-Uhlenbeck, le modèle CIR ou encore le modèle de Black Scholes fractionnaire.
   	\subsubsection{Présentation sous la forme différentielle et sous la forme intégrale}
		L'équation différentielle stochastique dirigés par un mouvement brownien fractionnaire est donnée par :  \[  dY_t = f(Y_t, t)dt + g(Y_t, t)dB_t^H  \] 
		Où $B_t^H$ est un mouvement brownien fractionnaire standard. \newline 
		Une présentation alternative, sous forme d'intégrale :
		  \[ Y_t = Y_s + \int_{s}^{t}f(Y_u, u)du + \int_{s}^{t}g(Y_u, u)dB_u^H \tag{1} \] 
		Pour le cas où $H=\frac{1}{2}$ l'intégrale stochastique de l'équation (1) est une martingale locale, ainsi le le processus $Y_t$ est, au minimum, une semi-martingale, ainsi la formule d'Itô est applicable. \newline
		Cependant, dans le cas où $H \ne \frac{1}{2}$ il est nécessaire de discuter les cas en fonction de H : 
		\begin{itemize}
    			\item Dans le cas où $H > \frac{1}{2}$ sous certaines conditions de régularité, on peut définir l'intégrale de Stieltjes.
   			\item Dans le cas où $H \in ]\frac{1}{4};\frac{1}{2}[$ on pourrait définir une notion de chemins rugueux pour définir des intégrales stochastiques fractionnaires.
		\end{itemize}
		
	\subsubsection{Schéma numérique de Milstein}
		Une EDS ne possède pas souvent une solution théorique explicite (formule fermée), et il est donc parfois nécessaire d'avoir recours à des Schémas numérique de calcul, dans le cas où nous avons une EDS dirigée par un mouvement brownien standard, il peut être judicieux d'utiliser un Schéma numérique d'Euler. \newline \newline
		Le Schéma de Milstein est valable pour tout $H \in ]0;1[$, sa formule est donnée par :
		\begin{multline}
 			\chi(t_{i+1}) =  \chi(t_i) + f(\chi(t_i), ti)\Delta t + g(\chi(t_i), t_i)\Delta B_i^H \\ + \frac{1}{2}g(\chi(t_i), t_i)g\prime(\chi(t_i), ti) [(\Delta B_i^H)^2 - (t_{i+1}^{2H} - t_i^{2H})] + \epsilon_i
		\end{multline}
		$\epsilon_i$ représente un résidu (différence entre la solution théorique et la solution numérique donnée par la solution numérique de Milstein) \newline \newline
		On peut donner une version plus courte et plus pratique de la formule, en partant du principe que : 
		$t_{i+1}^{2H} = (t_i + \Delta t)  \approx t_i^{2H} + 2Ht_i^{2H-1}\Delta t$ \newline \newline
		Ce qui nous donne donc : 
		\begin{multline}
 			\chi(t_{i+1}) =  \chi(t_i) + f(\chi(t_i), ti) - g(\chi(t_i), t_i)g\prime(\chi(t_i), t_i)Ht_i^{2H-1})\Delta t  \\ + g(\chi(t_i), t_i)\Delta B_i^H  + \frac{1}{2}g(\chi(t_i), t_i)g\prime(\chi(t_i), t_i)(\Delta B_i^H)^2 + \epsilon_i\prime
		\end{multline}
		$\epsilon_i\prime$ représente le résidu, et est supérieur à $\epsilon_i$ car l'approximation de $t_{i+1}^{2H}$ accroît cette marge d'approximation.
		
	\subsection{Modèles et Processus construits sur le mouvement brownien fractionnaire}
		\subsubsection{Linéarisation d'équations différentielles stochastiques unidimensionnelle}
			Pour pouvoir avancer dans la suite de notre étude (pour pouvoir donner une solution exacte du Processus d'Ornstein-Uhlenbeck fractionnaire) nous avons besoin d'introduire, suivant la présentation de [UD07], une méthode de construction de solutions théoriques. \newline
			\paragraph{Théorème 1} Posons : 
			\[ dx = f(x,t)dt + g(x,t)dB^H(t)  \tag{1} \] 
			Où $x(0) = x_0, t \geq 0, H \in (0,1)$ et $dB^H(t)$ l'incrément du mouvement brownien fractionnaire $B^H(t)$ , avec f et g, deux fonctions supposées assez régulières, l'équation (1) peut s'écrire sous la forme linéaire suivante : 
			\[ dy = (a(t)y + b(t)dt)dt + (c(t)y + e(t))dB^H(t) \tag{2} \] 
			Via la transformation inverse $y=h(x,t)$ avec sa dérivée partielle par rapport à $x$ non-nulle si et seulement si on a :
			\[ \frac{\partial}{\partial x}\left(\frac{\frac{\partial}{\partial x}(g(x, t)) \frac{\partial}{\partial x}(g(x,t)L)}{\frac{\partial}{\partial x}(g(x, t)L)}\right) = 0  \] 
			Avec $L = \frac{\partial}{\partial x}\left(\frac{1}{g}\right) +  \frac{\partial}{\partial x} \left(\frac{f}{g} - Ht^{2H - 1} \frac{\partial g}{\partial x}\right) $ \newline
			Les étapes de la linéarisation sont les suivantes, et dépendent de $c(t)$ de l'équation (2) : 
			\begin{itemize}
    				\item Si c(t) = 0 alors on a :
					\[ h(x,t) = \left(    \int \alpha(t)dt    \right)^{-1}    \left(    \int \frac{1}{g(x,t)}dx   \right) \] 	
					La condition $\frac{\partial}{\partial x}\left(\frac{\frac{\partial}{\partial x}(g(x, t)) \frac{\partial}{\partial x}(g(x,t)L)}{\frac{\partial}{\partial x}(g(x, t)L)}\right) = 0$ est bien respectée et l'équation différentielle stochastique linéaire s'écrit donc : 
					\[ dy = \beta(t)e(t)dt + e(t)dB^H\] \newline
					avec $e(t) = \left(    \int \alpha(t)dt    \right)^{-1}$, $\alpha$ et $\beta$ sont définies par identification dans l'égalité : 
					\[ \alpha(t)y + \beta(t) = \int \frac{\partial}{\partial t} \left( \frac{1}{g}\right) + \frac{f}{g} - Ht^{2H-1} \frac{\partial g}{\partial x} \] 
					
   				\item Si $c(t) \ne 0$ alors on a :
					\[ h(x,t) = e^{-M(t) \int \frac{1}{g(x,t)}dx}\] \newline
					Avec $M(t) = \frac{\frac{\partial}{\partial x}(g(x, t)) \frac{\partial}{\partial x}(g(x,t)L)}{\frac{\partial}{\partial x}(g(x, t)L)} $
			\end{itemize}
			Nous avons maintenant une forme linéaire de notre équation, il faut maintenant la résoudre en utilisant la méthode du facteur intégrant, définit comme suit : 
			\paragraph{Définition 1}
				La fonction $F=F(t, B^H)$ vérifiant $d(Fy) = D_1(t,B^H)dt + D_2(t,B^H)dB^H$ est un facteur intégrant pour des équations différentielles stochastiques linéaires unidimensionnelles, on peut donc écrire : 
				\[ F(t, W^H) = exp \left(- \int_{}^{t} c(s)dB^H(s) + H \int_{}^{t}s^{2H - 1}c^2(s)ds - \int_{}^{t} a(s)ds \right) \] 
				Par suite, on a donc la solution de l'équation différentielle stochastique linéaire suivante :
				\[ y(t) = \frac{1}{F}\left( \int_{}^{t} F(s)b(s)ds - 2H \int_{}^{s} s^{2H-1}F(s)c(s)e(s)ds + \int_{}^{t}F(s)e(s)dB^H(s)+y_0 \right) \] 
				
		\subsubsection{Présentation du processus d'Ornstein-Uhlenbeck fractionnaire}
			Le processus de Ornstein-Ulhenbeck associé au mouvement brownien fractionnaire associé est donné par l'équation suivante : 
      	 		\[ dx = \theta(t)(\mu (t)(t)-x)dt + \sigma (t) dB^H_t\] 
			Ce processus est souvent utilisé pour modéliser des taux en finance.
			
		
		\subsubsection{Solution théorique excacte}
		Pour donner une solution théorique exacte on utilise la méthode du facteur intégrant précédemment introduit (cf section 1.6.1) dont l'expression est la suivante : $F(t) = e^{\int_{}^{t}\theta(s)ds}$ \newline
		On a donc la solution suivante :
		\[ y(t) = e^{\int_{}^{t} \theta(s)ds}\left(  \int_{}^{t} e^{\int \theta(s)ds} \theta(s)\mu(s)ds + \int_{}^{t} e^{\int \theta(s)ds} \sigma(s)dB^H(s) +y_0 \right) \tag{3}\] \newline
		 Il s'agit maintenant de donner une solution numérique à (3). 
		Pour cela, on utilise le schéma de Milstein et on a donc, pour le paramètre de Hurst H quelconque : 
		\[ X_{j+1} = X_j + \theta(\mu - X_j)\Delta t + \sigma \Delta B_j^H \] 
		
		\paragraph{Quelques simulations du processus d'Ornstein-Uhlenbeck fractionnaire \dots}
		 \phantom{a}
		
		\begin{figure}[h!]
  			\includegraphics[width=\linewidth, height=80mm]{SimulationspO-U-H.png}
  			\caption{Simulation du processus d'Ornstein-Uhlenbeck pour différentes valeurs de H}
 			\label{fig:Simulation du processus d'Ornstein-Uhlenbeck pour différentes valeurs de H}
		\end{figure}
		\newpage
		
		
		\subsubsection{Modèle de Cox-Ingersoll-Ross fractionnaire}
			Le modèle CIR dirigé par un mouvement brownien fractionnaire est défini par la dynamique suivante : 
			\[ dx = \gamma(t)(\theta(t)-x)dt + \sigma(t)\sqrt{x}dB^H(t) \]
			Ce modèle est utilisé afin de modéliser l'évolution de taux d'intérêts \newline
			On a : $f(x,t) = \gamma(t)(\theta(t)-x)$, $g(x,t) = \sigma(t)\sqrt{x}$. On a donc $L = \frac{1}{2} \frac{\sigma(t)}{\sqrt{x}} + \frac{\sigma(t)}{\sigma(t)} \frac{-1}{2}x^{-\frac{3}{2}}\theta(t) - \frac{1}{2}x^{-\frac{1}{2}} -Ht^{2H-1}\frac{-\sigma(t)}{4}x^{-\frac{3}{2}}$ \newline \newline
			Puis on a : $\frac{\partial}{\partial x} g(x,t)L = x^{-2}(\frac{1}{2}\gamma(t)\theta(t) -  \frac{\sigma(t)^2}{4} Ht^{2H-1}$ \newline \newline
			On remarque que pour $\sigma(t) = \sqrt{\frac{2\gamma(t)\theta(t)}{Ht^{2H-1}}},$ $\frac{\partial}{\partial x} g(x,t)L$ s'annule et donc d'après la méthode décrite dans 1.6.1 on peut écrire l'équation différentielle stochastique sous forme linéaire.
			Nous devons d'abord déterminer $\alpha(t)y$ comme décrit à la section 1.6.1 :
			\[ \alpha(t)y+\beta(t) = 2x^{\frac{1}{2}} \frac{\partial}{\partial t} \frac{1}{\sigma(t)} + \frac{\gamma(t)}{\sigma(t)} (\theta(t)x^{-\frac{1}{2}} - x^\frac{1}{2}) - \frac{1}{2}Ht^{2H-1}\sigma(t)x^{-\frac{1}{2}} \]
			On a donc : 
			\[ \alpha(t)y+\beta(t) = x^\frac{1}{2}(-\frac{\gamma(t)}{\sigma(t)} + 2  \frac{\partial}{\partial t}\frac{1}{\sigma(t)})  \]
			Or $\beta(t)$ vaut 0 on a donc :
			\[ (\int \alpha(t)dt)^\prime + (\frac{\gamma(t)}{2} + \sigma(t) \frac{\partial}{\partial t}\frac{1}{\sigma(t)}) \int \alpha(t)dt  = 0 \]
			On peut donc égaliser : 
			\[ \int \alpha(t)dt = e^{-\int \frac{\gamma(t)}{2}dt } \sigma(t)^{-1}\]	
			On obtient finalement : 
			\[ y = 2e^{\int \frac{\gamma(t)}{2}dt}\sqrt{x} \]
			On peut donc écrire l'équation différentielle stochastique linéaire : 
			\[ dy = e^{\int \frac{\gamma(t)}{2}}\sigma(t)dB^H(t) \]			
			On obtient la solution en intégrant :
			\[ y(t) = \int_{0}^{t}e^{\int \frac{\gamma(t)}{2}}\sigma(t)dB^H(t) + K \] \newline
			Enfin, on obtient donc la solution du modèle CIR : 
			\[ x(t) = \left(\frac{1}{2}exp\left(-\int \frac{1}{2} \gamma(t)dt \right) \left(\int_{0}^{t}exp\left(\int \frac{1}{2}\gamma(t)dt \right) \sigma(t)dB^H(t) + \sqrt{x(0)} \right) \right)^2  \]
			On utilise maintenant un schéma numérique de Milstein afin de donner une solution numérique de l'expression plus haut, on a donc : 
			\[ X_{j+1} = X_j + (\gamma(\theta-X_j) - \frac{1}{2}H\sigma^2t_j^{2H-1})\Delta t + \sigma\sqrt{(X_j)} \Delta B_j^H + \frac{1}{4}\sigma^2(\Delta B^H_j)^2\]
			
		\paragraph{Quelques simulations du modèle CIR fractionnaire \dots } 
		 \phantom{a}
		\begin{figure}[h!]
  			\includegraphics[width=\linewidth, height=80mm]{SimulationsCIRfrac.png}
  			\caption{Simulations du modèle CIR fractionnaire pour différentes valeurs de H}
 			\label{fig:Simulations du modèle CIR fractionnaire pour différentes valeurs de H}
		\end{figure}
		\newpage
			
			
		\subsubsection{Le modèle de Black-Scholes fractionnaire}
			Le modèle de Black-Scholes existe aussi en version fractionnaire, la partie aléatoire est modélisée par un mouvement brownien fractionnaire. \newline
			Cependant, ce modèle n'est pas très utilisé en pratiqué car il écarte l'hypothèse d'absence d'opportunités d'arbitrages
		\subsubsection{Modélisation d'un actif sous-jacent sous Black-Scholes fractionnaire}
			La seule différence faite avec le modèle de Black Scholes classique se trouve au niveau du mouvement brownien utilisé, en effet on a : 
			\[ dS_t = \mu S_t + \sigma S_tdB^H_t\] 
			Afin de simuler le modèle de Black-Scholes fractionnaire nous utiliserons le schéma de Milstein, on a donc :
			\[  S_{t_{i+1}} = S_{t_i} + rS_{t_i}\Delta t - \sigma^2 S_{t_i}Ht_i^{2H-1}\Delta t + \sigma S_{t_i} \Delta B_i^H + \frac{1}{2}\sigma^2S_{t_i}(\Delta B_i^H)^2 \] 
			
		\paragraph{Quelques simulations du modèle de Black-Scholes fractionnaire \dots}
		\phantom{a}
		\begin{figure}[h!]
  			\includegraphics[width=\linewidth, height=80mm]{SimulationsBSfrac.png}
  			\caption{Simulations du modèle de Black-Scholes fractionnaire pour différentes valeurs de H}
 			\label{fig:Simulations du modèle Black-Scholes fractionnaire pour différentes valeurs de H}
		\end{figure}
		\newpage
		
		\subsection{Modèle Rough Heston}
			\subsubsection{Modèle de Heston classique}
				Le Modèle de Heston standard est un modèle à volatilité stochastique très utilisé, il est définit par la dynamique de prix suivante : 
				\[ dS_t = S_t\sqrt{V_t}dW_t\] 
				De plus, il est régit par un processus de variance décrit par l'équation différentielle suivante : 
				\[ dV_t = \lambda(\theta - V_t)dt + \lambda \nu \sqrt{V_t}dB_t\] avec 	$\langle dW_t, dB_t \rangle = \rho dt$, avec $\rho \in [-1;1]$ \newline
				$W_t$ et $B_t$  sont deux mouvement brownien standard
			\subsubsection{Modèle de Heston dirigé par un mouvement brownien fractionnaire}
				Dans le modèle de Heston fractionnaire, le processus de volatilité est dirigé par un mouvement brownien fractionnaire, comme vu précédemment, selon Gatheral et Rosenbaum la log-volatilité se comporte comme un mouvement brownien fractionnaire de paramètre de Hurst $H \in ]0;1[$. \newline \newline
				De plus, cette modélisation (par un mBF) ne modifie en rien les propriétés statistiques de la volatilité.Ainsi l'équation différentielles sont données par : 
				
				\[ dS_t = S_t\sqrt{V_t}dW_t^H\]
				Où : $B_t^H$ et $W_t^H$ sont deux mouvements browniens fractionnaires 
				\newline\newline
				La solution donnée pour $dV_t$ [Cours Rosenbaumm] est la suivante : 
				\[  V_t = V_0 + \frac{1}{\Gamma(\alpha)}\int_{0}^{t}(t-s)^{\alpha-1}\lambda(\theta - V_s)ds + \frac{\lambda \nu}{\Gamma(\alpha)}\int_{0}^{t}(t-s)^{\alpha - 1}\sqrt{V_t}dB_s\]				
				Selon Alos, Fukasawa et Bayer, de tels modèles permettent de bien reproduire le comportement de la surface de volatilité implicite, en particulier la dépendance de la volatilité par rapport au strike (à la monnaie). \newline \newline
				Dans le modèle de Heston classique la fonction caractéristique des log-prix ($X_t$) est donnée par : $X_t = log(S_t/S_0)$, en particulier cette équation satisfait : 
				\[ \mathbb{E}[e^{iaX_t}]  = exp(g(a,t) + V_0h(a, t))\]
				Où $h$ est la solution de l'équation de Riccati suivante : 
				\[ \partial_t h = \frac{1}{2}(-a^2 - ia) + \lambda(ia\rho \nu - 1)h(a,s) +  \frac{(\lambda \nu)^2}{2}h^2(a,s) \tag{1}, \;\;\; h(a,0) = 0 \]
				\[ L(t,\lambda, V_t, S_t) = \mathbb{E}[e^{i\lambda log(S_T)} \lvert F_t]\] \newline
				Avec $F_t$ la filtration engendrée par les browniens dirigeants la volatilité et le prix. \newline
				$L$ étant une martingale, on a en appliquant la formule d'Itô l'équation suivante (les termes en $dt$ doivent être nuls) : 
				\[  \partial_t L + \lambda (\theta - V)\partial_V L + \frac{1}{2} (\lambda v)^2 V \partial^2_{VV}L + \frac{1}{2}S^2V \partial_SL  + \rho v \lambda SV \partial^2_{SV}L  = 0\] 		
			\subsubsection{Construction du modèle Rough Heston}
				Nous reprendrons la construction du modèle Rough Heston tel que Rosenbaum l'a construit.
				On voudrait un modèle tick-by-tick qui prend en compte l'aspect fractionnaire des marchés et qui reproduit les propriétés importantes du modèle de Heston :
				\begin{itemize}
    					\item L'effet de levier : La capacité d'amplification de la capacité d'investissement.
   					\item Prendre en compte l'effet d'asymétrie (Kurtosis), de queue lourde.
					\item La modélisation permet de reproduire la dynamique du volatility smile (Dépendance de la volatilité implicite par rapport au Strike $K$), comme prévu par le modèle de Black-Scholes on remarque que la volatilité implicite n'est pas constante mais dépendante du strike $K$
				\phantom{a}
				\begin{figure}[h!]
  					\includegraphics[width=\linewidth, height=80mm]{volatilitySmile.jpeg}
  					\caption{Exemple de volatility smile}
 					\label{fig:}
					\end{figure}
				\newpage		
				\end{itemize}
				En pratique le modèle de Heston, et donc le modèle Rough Heston sont utilisés afin de pricer des options exotiques, mais cette modélisation tend à créer une divergence entre le pricing d'options vanilles (faussés) et la modélisation de ces derniers par un modèle à volatilité constante/déterministe (plus réel).
		 		Cependant un problème subsiste concernant le modèle Rough Heston, ce dernier n'est plus Markovien on ne peut donc plus utiliser l'approche classique utilisé dans les modèles à EDS dirigés par un mouvement brownien standard, c'est à dire utiliser la formule d'Itô, nous ne rentrerons pas dans les détails sur ce point.						\newline\newline
				L'approche qui a été utilisé par El Euch et Rosenbaum consiste à regarder des fonctionnelles de processus de Hawkes qui convergent vers le modèle de Rough Heston, l'idée étant que l'on connaît les fonctions caractéristiques des processus de Hawkes, on peut donc facilement à la limite et donc obtenir des formules pour le 					modèle Rough Heston.
				La fonction caractéristique du log-prix dans le modèle Rough Heston montre la même structure que la fonction caractéristique du modèle de Heston classique.\newline
				A l'exception que l'équation de Riccati (1) est remplacée par une équation de Riccati fractionnaire où une dérivée fractionnaire est utilisé à la place d'une dérivée classique, on a plus précisément : 
				\[ \mathbb{E}[e^{iaX_t}] = exp(g_1(a,t) + V_0g_2(a,t)) \] \newline
				Avec : 
				\[ g_1(a, t) = \theta \lambda\int_{0}^{t}h(a,s)ds, g_2(a,t) = I^{1-\alpha}h(a,t), \] \newline
				De plus, $h(a,.)$ est une solution de l'équation de Riccati fractionnaire suivant : 
				\[ D^\alpha h(a,t) = \frac{1}{2}(-a^2-ia)+\lambda(ia\rho v - 1)h(a,t) + \frac{(\lambda v)^2}{2}h^2(a,t),\;\;\; I^{1-\alpha}h(a,0) = 0,  \] \newline
				Avec $D^\alpha$ et $I^{1-\alpha}$ la dérivée fractionnaire et l'opérateur intégrant définit comme suit : \newline
				\paragraph{Intégrale fractionnaire}
					Pour $f : \mathbb{R} \rightarrow \mathbb{R}$ appartenant à $L^1(\mathbb{R})$ on peut définir l'opérateur intégrale fractionnaire de Riemann-Liouville d'ordre $\alpha > 0$ par : 
					\[ I^{1-\alpha} f(x) = \frac{1}{\Gamma(1-\alpha)}\int_{0}^{x}\frac{f(t)}{(x-t)^\alpha}dt \] \newline
				\paragraph{Dérivée fractionnaire}
					Dans les mêmes conditions que pour l'intégrale fractionnaire, on peut définir l'opérateur de dérivée fractionnaire comme suit : 
					\[ D^\alpha f(x) = \frac{d}{dx}f^{1-\alpha}f(x) \] \newpage
				Il convient de remarquer que lorsque $\alpha = 1$, les résultats coïncident avec les résultats du modèle de Heston classique, cependant lorsque $\alpha < 1$ la solution de l'équation de Riccati ne peuvent être établis explicitement, cependant, elles peuvent être facilement numériquement résolue. \newline
				Finalement, la loi du processus $(S_t^{t_0}, V_t^{t_0})_{t \geq 0} = (S_{t+t_0}, V_{t+t_0})_{t \geq 0}$ est la loi du modèle Rough Heston dont les dynamiques de volatilités et de prix sont les suivants : 
				\[ dS_t^{t_0} = S_t^{t_0}\sqrt{V_t^{t_0}}dW_t^{t_0}; \;\;\; S_0^{t_0} = S_{t_0} \]  \newline
				\[ V_t^{t_0} = V_{t_0} + \frac{1}{\Gamma(\alpha)}\int_{0}^{t}(t-s)^{\alpha-1}\lambda(\theta^{t_0}(s)-V_s^{t_0})ds+\frac{\lambda\nu}{\Gamma(\alpha)}\int_{0}^{t}(t-s)^{\alpha-1}\sqrt{V_s^{t_0}}dB_s^{t_0}\] \newline
				Avec : $(W_t^{t_0}, B_t^{t_0}) = (W_{t_0+t}+W_{t_0}, B_{t_0+t}+B_{t_0})$ et $\theta^{t_0}$ est un processus $F_t-$mesurable, dépendant de $(V_u)_{0\leq u \leq t_0}.$ \newline
				On peut aussi donner une forme généralisée du modèle Rough Heston :
				\[ dS_t = S_t\sqrt{V_t}dW_t \] \newline
				\[ V_t = V_0 + \frac{1}{\Gamma (\alpha)} \int_{0}^{t} (t-s)^{\alpha-1}\lambda(\theta^0(s)-V_s)ds + \frac{\lambda \nu}{\Gamma(\alpha)}\int_{0}^{t}(t-s)^{\alpha - 1}\sqrt{V_s}dB_s, \]  \newline
				Avec $\langle dW_t, dB_t \rangle = \rho dt,  \;\;\; \alpha \in (1/2, 1).$
				
				\subsubsection{Démonstration de la loi du couple ($S_t^{t_0}, V_t^{t_0}$) conditionnellement à $F_t$ }
				Comme l'on démontré El Euch et Rosenbaum, il est nécessaire d'utiliser le théorème de Fubini stochastique, l'objectif est de montrer que $I^{1-\alpha}V	$ est une semi-martingale pour $t > 0$, posons : 
				\[  (I^{1-\alpha}V)_t = V_0\int_{0}^{t}\frac{s^{-\alpha}}{\Gamma(1-\alpha}ds + \int_{0}^{t} \lambda(\theta^0(s) - V_s)ds + \int_{0}^{t}\nu\sqrt{V_s}dB_s \]
				On a donc : 
				\[ \frac{1}{\Gamma(1-\alpha)} \int_{0}^{t+t_0}(t+t_0 - u)^{-\alpha} V_udu \]
				qui vaut :
				\[ \frac{1}{\Gamma(1-\alpha)}\int_{0}^{t_0}(t_0-u)^{-\alpha} V_udu + V_0 \int_{t_0}^{t+t_0}\frac{1}{\Gamma(1-\alpha)}u^{-\alpha}du + \int_{t_0}^{t+t_0} \lambda(\theta^0(u) - V_u)du + \int_{t_0}^{t+t_0}\nu\sqrt{V_u}dB_u \]
				En utilisant un changement de variable $t^\prime = t-t_0$, on a :
				\[ \frac{1}{\Gamma(1-\alpha)} \int_{0}^{t_0}(t_0-u)^{-\alpha}V_udu + V_0\int_{0}^{t} \frac{1}{\Gamma(1-\alpha)}(t_0+u)^{-\alpha}du + \int_{0}^{t}\lambda(\theta^0(u+t_0)-V_u^{t_0})du +  \int_{0}^{t} \nu \sqrt{V_u^{t_0}}dB_u^{t_0} \]
				Avec $(B_t^{t_0})_{t \geq 0} = (B_{t+t_0} - B_{t_0})$ est un mouvement brownien indépendant de $F_{t_0}$
				De plus, on a :
				\[ I^{1-\alpha}V_t^{t_0}  = \frac{1}{\Gamma(1-\alpha)} \int_{0}^{t}(t-u)^{-\alpha}V_u^{t_0}du = \frac{1}{\Gamma(1-\alpha)} \int_{t_0}^{t+t_0}(t+t_0-u)^{-\alpha}V_udu \]
				qui vaut :
				\[ \frac{1}{\Gamma(1-\alpha)} \int_{0}^{t+t_0}(t + t_0-u)^{-\alpha}V_u^{t_0}du - \frac{1}{\Gamma(1-\alpha)} \int_{t_0}^{t_0}(t+t_0-u)^{-\alpha}V_udu \]
				Et,
				\[ \frac{1}{\Gamma(1-\alpha)} ((t_0 - u)^{-\alpha} - (t + t_0 - u)^{-\alpha}) = \frac{\alpha}{\Gamma(1-\alpha)}\int_{0}^{t}(t+t_0 - u)^{-1-\alpha}dv  \]
				Puis on dérive :
				\begin{multline}
					I^{1-\alpha}V_t^{t_0} = \frac{1}{\Gamma(1-\alpha}\int_{0}^{t_0} \int_{0}^{t}(t_0-u+v)^{-1-\alpha}dvV_udu + \int_{0}^{t}(t_0+u){-\alpha}duV_0 \\ + \int_{0}^{t}\lambda(\theta^0(u+t_0) - V_u^{t_0}du + \int_{0}^{t} \nu \sqrt{V_u^{t_0}}dB_u^{t_0}
				\end{multline}
				On peut encore réécrire cela comme suit : 
				\[ V_{t_0} \frac{t^{1-\alpha}}{(1-\alpha)\Gamma(1-\alpha)} + \int_{0}^{t} \lambda(\theta^{t_0}(u) - V_u^{t_0})du + \int_{0}^{t}\nu\sqrt{V_u^{t_0}}dB_u^{t_0}, \]
				Où ($\theta^{t_0}(u))_{u \geq 0}$) est une fonction $F_{t_0}-$mesurable et définit par :  
				\[ \theta^{t_0}(u) = \theta^0(t_0+u) + \frac{\alpha}{\lambda \Gamma(1-\alpha)} \int_{0}^{t_0}(t_0 - v + u )^{-1-\alpha}(V_u-V_{t_0}dv + \frac{(u+t_0)^{-\alpha}}{\lambda\Gamma(1-\alpha)}(V_0 - V_{t_0}) \] \newline \newline
				
				\paragraph{Propriétés de $\theta^{t_0}$}
					$\theta^{t_0}$ est continue sur $\mathbb{R}_+^+$ et. de plus, pour tout $u >0$ on a :
					\[ \theta^{t_0}(u) = \theta^0(t_0 + u) + \frac{\alpha}{\lambda\Gamma(1-\alpha)} \int_{0}^{t_0}(t_0 - v + u)^{-1 - \alpha}V_udv + \frac{1}{\lambda \Gamma(1-\alpha)}(V_0(u+t_0)^{-\alpha} - V_{t_0}u^{-\alpha}) \]
					V est un processus non-négatif et $\theta^0$.
					Finalement, pour $\epsilon > 0$, V est un processus $\alpha - 1/2 - \epsilon$ Hölder-continue, autrement dit pour chaque $\omega \in \Omega$, et avec $c_\epsilon(\omega)$ tel que pour $x, y \in [0, t_0]$ on a :
					\[ \lvert V_x - V_y \rvert \leq c_\epsilon(\omega) \lvert x - y \rvert^{\alpha - 1/2 - \epsilon} \]
					Ensuite, par intégration par parties on a, pour tout u $\in (0, t_0]$:
					\[ \lvert \int_{0}^{t_0}(t_0 - v + u)^{-1-\alpha}(V_v - V_{t_0})dv \rvert \leq c_\epsilon(\omega)\int_{0}{t_0}(t_0 - v + u)^{-1-\alpha}(t_0 - v)^{\alpha - 1/2 - \epsilon}dv  \]
					\[ = c_\epsilon(\omega)u^{-1/2 - \epsilon} \int_{0}^{t_0/u}(x+1)^{-1-\alpha}x^{\alpha-1/2-\epsilon}dx \]
					\[ \leq c_\epsilon(\omega)u^{-1/2 - \epsilon} \int_{0}^{\infty}(x+1)^{-1-\alpha}x^{\alpha - 1/2 - \epsilon}dx\]
					
				Pour finir nous avons : 
					\[ \int_{0}^{t}V_s^{t_0}ds = \frac{1}{\Gamma{\alpha}} \int{0}{t}(t-s)^{\alpha - 1}I^{1-\alpha}V_s^{t_0}ds \]
				qui vaut :
					\[ V_{t_0}t + \frac{1}{\alpha}\int_{0}^{t}\int_{0}^{s}(s-u)^{\alpha - 1}\lambda(\theta^{t_0}(u) - V_u^{t_0})duds + \frac{1}{\Gamma(\alpha)} \int_{0}^{t}\int_{0}^{s}(s-u)^{\alpha - 1}\nu\sqrt{V_u^{t_0}}dB_u^{t_0}ds \] \newline
				On conclut, en différentiant l'équation au dessus que la dynamique de ($S^{t_0}, V^{t_0}$) est donnée par : 
					\[ S_t^{t_0} = S_{t_0}exp \left(  \int_{0}^{t}\sqrt{V_u^{t_0}}dW_u^{t_0} - \frac{1}{2} \int_{0}^{t}V_u^{t_0}du \right) \] \newline
					\[ V_t^{t_0} = V_{t_0} + \frac{1}{\Gamma(\alpha)}\int_{0}^{t}(t-u)^{\alpha - 1}\lambda(\theta^{t_0}(u) - V_u^{t_0})du + \frac{1}{\Gamma(\alpha)} \int_{0}^{t}(t-u)^{\alpha - 1}\nu\sqrt{V_u^{t_0}}dB_u^{t_0} \] \newline
					Où : $(W_t^{t_0})_{t \geq 0} = (W_{t+t_0} - W_{t_0})_{t \leq 0}$ est un mouvement brownien indépendant de $F_{t_0}$ et tel que : $\langle dW_t, dB_t \rangle = \rho$
					
				
				
				\subsubsection{Schéma numérique de calcul de la fonction caractéristique du log-prix dans le modèle Rough Heston}
					Comme l'on expliqué El Euch et Rosenbaum, l'objectif est de calculer numériquement la fonction caractéristique, nous avons : 
					\[ D^\alpha h(a,t) = \frac{1}{2}(-a^2-ia)+\lambda(ia\rho v - 1)h(a,t) + \frac{(\lambda v)^2}{2}h^2(a,t),\;\;\; I^{1-\alpha}h(a,0) = 0 \tag{1},  \] \newline
					Avec : 
					\[ F(a,x) = \frac{1}{2}(-a^2 - ia) + \lambda(ia\rho \nu - 1)x +  \frac{(\lambda \nu)^2}{2}x^2, \;\;\; h(a,0) = 0 \] \newline
					Bien qu'il existe plusieurs schéma numérique permettant de résoudre numériquement (1), l'approche utilisée est basée sur le fait que (1) implique l'équation de Volterra suivante : 
					\[ h(a,t) = \frac{1}{\Gamma(\alpha)}\int_{0}^{t} (t-s)^{\alpha-1}F(a,h(a,s))ds\] \newline
					Puis de développer un schéma nuémrique basé sur l'équation précédente.
					En utilisant la méthode fractionaire Adams, où l'idée est la suivante : \newline
					Posons : $g(a,t) = F(a,h(a,t))$, puis discrétisons la fonction $h$ en utilisant un pas $\Delta$ tel que $(t_k = k\Delta)$, l'estimation de $h$ : 
					\[h(a, t_{k+1}) = \frac{1}{\Gamma(\alpha)}\int_{0}^{t_{k+1}}(t_{k+1}-s)^{\alpha-1}g(a,s)ds \] \newline
					Est donnée par :
					\[ \frac{1}{\Gamma(\alpha)}\int_{0}^{t_{k+1}}(t_{k+1}-s)^{\alpha-1}\hat{g}(a,s)ds \] \newline
					Avec : 
					\[ \hat{g}(a,t) = \frac{t_{j+1} - t}{t_{j+1} - t_j} \hat{g}(a,t_{j})+  \frac{t - t_j}{t_{j+1} - t_j} \hat{g}(a,t_{j+1}), \;\;\; t \in [t_j, t_{j+1}], \;\;\; 0 \leq j \leq  k \] \newline
					Cela nous amène au schéma suivant : 
					\[ \hat{h}(a, t_{k+1} = \sum_{0 \leq j \leq  k} a_{j, k+1}F(a, \hat{h}(a, t_j)) +   a_{k+1, k+1}F(a, \hat{h}(a, t_{k+1}))\] \newline
					Où :
					\[ a_{0, k+1} = \frac{\Delta^\alpha}{\Gamma(\alpha+2)} (k^{\alpha +1} - (k-\alpha)(k+1)^\alpha)\] \newline
					\[  a_{j, k+1} =  \frac{\Delta^\alpha}{\Gamma(\alpha+2)} ((k-j+2)^{\alpha + 1} + (k-j)^{\alpha+1} + (k-j)^{\alpha + 1} - 2(k-j+1)^{\alpha + 1}), \;\;\; 1 \leq j \leq k \] \newline
					De plus, 
					\[a_{k+1, k+1} = \frac{\Delta^\alpha}{\Gamma(\alpha+2)}  \] \newline
					On calcule ensuite une pre-estimation de $\hat{h}(a, t_{k+1})$ basée sur une somme de Riemann 
					\[ \hat{h}^P(a, t_{k+1})  = \frac{1}{\Gamma(\alpha)} \int_{0}^{t}(t-s)^{\alpha - 1}\tilde{g}(a,s)ds,\] \newline
					Où : 
					\[ \tilde{g}(a, t) = \tilde{g}(a, t_j), \;\;\; t \in [t_j, t_{j_+1}), \;\;\; 0 \leq j \leq k  \] \newline
					Donc on a :
					\[ \tilde{h}^P(a, t_{k+1})  =  \sum_{0 \leq j \leq  k} b_{j, k+1}F(a, \hat{h}(a, t_j))\] \newline
					Avec :
					\[ b_{j, k+1} = \frac{\Delta^\alpha}{\Gamma(\alpha + 1)}((k-j+1)^\alpha - (k-j)^\alpha), \;\;\; 0 \leq j \leq k \] \newline
					Et finalement on a donc : 
					\[ \hat{h}(a, t_{k+1}) = \sum_{0 \leq j \leq  k} a_{j, k+1}F(a, \hat{h}(a, t_j)) + a_{k+1, k+1}F(a, \hat{h}^P(a, t_j)), \;\;\; \hat{h}(a, 0) = 0,  \] \newline
					En théorie, nous devons avoir les résultats de convergence suivants, pour $ t  > 0$ et $a \in \mathbb{R}$ : 
					\[ \max_{t_j \in [0,t]} \lvert \hat{h}(a, t_j) - h(a, t_j) \rvert = o(\Delta)  \] \newline
					et, $\epsilon > 0$
					\[ \max_{t_j \in [\epsilon,t]} \lvert \hat{h}(a, t_j) - h(a, t_j) \rvert = o(\Delta^{2-\alpha})  \]
				
			\subsection{Le modèle rough SABR}
				\subsubsection{Le modèle SABR classique}
					Le modèle SABR (stochastic-alpha-bêta-rho) est un modèle à volatilité stochastique utilisé principalement sur les marchés FOREX et les marchés à taux d'intérêt, l'intérêt de ce modèle est que l'on a explicitement une formule fermée pour la volatilité implicite, sa la dynamique est donnée par : 
					\[ dF_t = \sigma_t(F_t)^{\beta}dW_t  \] 
					\[ d\sigma_t = \alpha \sigma_t dB_t  \] \newline
					Avec $f_t : $ Forward, et 
					$\langle dW_t, dB_t \rangle = \rho dt$, \newline
					La formule de volatilité implicite est donnée par : 
					\[ \sigma_B(K,f) = \frac{\alpha \left( 1+ \left[  \frac{(1-\beta)^2}{24} \frac{\alpha^2}{(fK)^{(1-\beta}} + \frac{1}{4}\frac{\rho \beta v \alpha}{(fK)^{(1-\beta)/2}}   + \frac{2-3\rho^2}{24}v^2    \right]T      \right)}{(fK)^{(1-\beta)/2} \left[1+\frac{(1-\beta)^2}{24}ln^2 \frac{f}{K} + \frac{(1-\beta)^4}{1920}ln^4 \frac{f}{K}   \right]} \frac{z}{\chi(z)}\] 					\newline
					Avec : 
					\[ z = \frac{v}{\alpha}(fK)^{(1-\beta)/2}ln \frac{f}{K} \] \newline
					\[ \chi(z) = ln \left[  \frac{\sqrt{1-2\rho z+ z^2}+z - \rho}{1 - \rho}  \right]  \]
				\subsubsection{Le modèle SABR dirigé par un mouvement brownien fractionnaire}	
					
			\subsection{Le modèle RFSV}
				
					
										
					
	\newpage
   
   \section*{Conclusion}
   \addcontentsline{toc}{section}{Conclusion}
   
   
   \appendix
   
   \section{L'approche de Carr et Madan}
   	Carr et Madan propose une approche intéressant afin d'utiliser la transformé de Fourier rapide pour évaluer le prix des options. \newline
	En connaissant la fonction caractéristique, on peut, par des formules d'inversion de Fourier obtenir le prix d'un actif, notamment dans le cas que l'on évoquera ici, le prix d'un call en fonction de la fonction caractéristique. \newline
	L'idée étant que l'on connaît souvent la fonction caractéristique dans certains modèles, comme le modèle de Heston.
	\[ C_T(k) = e^{-rT} \mathbb{E}^\mathbb{Q}[(S_T - L)^+] \]
	\[ = e^{-rT}\int_{k}^{\infty} (e^x - e^k)q(x)dx \]
	On ne peut pas exprimer le Call en utilisant le log-strike car $C(k)$ tends vers $S_0$ quand $k \rightarrow -\infty$
	On a donc : 
	\[ C_T(k) = e^{-rT}\int_{-\infty}^{\infty}(e^x - e^{-\infty})q(x)dx, \]
	\[ = e^{-rT}\int_{-\infty}^{\infty}e^x q(x)dx, \]
	\[ = e^{-rT}\mathbb{E}^\mathbb{Q}[e^x] \]
	
	On en déduit, en utilisant la propriété des martingales $\mathbb{E}^{\mathbb{Q}[S_T]} = S_0e^{rT}$ on s'aperçoit que $lim_{k\rightarrow -\infty} C(k) = S_0$ ne converge pas vers 0.
	Cependant, $C(k)$ n'est pas $L^1$, une transformation de Fourier n'existe pas, cependant en introduisant un facteur exponentielle $e^{\alpha k}$ avec $\alpha > 0$, il est possible C(k) une fonction intégrable.
	\[ c_T(k) = e^{\alpha k}C_T(k), \]
	$c_T(k)$ est une fonction intégrable, car $\int_{-\infty}^{\infty} \lvert e^{\alpha k}C_T(k) \rvert dk < \infty $, est bien fini, pour $\alpha$ déterminé de manière optimale.
	La transformée de Fourier de $c(k)$ est donnée par :
	\[ \Psi(u) = \int_{-\infty}^{\infty} e^{iuk}c(k)dk, \]
	\[ = \int_{-\infty}^{\infty} e^{iuk} \int_{-\infty}^{\infty}e^{\alpha k}e^{rT}(e^x - e^k)^+q(x)dxdk, \]
	\[ = \int_{-\infty}^{\infty} e^{iuk} \int_{k}^{\infty}e^{\alpha k} e^{-rT}(e^x-e^k)q(x)dxdk, \]
	\[ = \int_{-\infty}^{\infty} e^{-rT}q(x) \left(\int_{-\infty}^{x}(e^x - e^k)e^{iuk}e^{\alpha k} dk \right)dx  \]	
	
	L'intégrale interne représente le payoff vaut :
	
	\[ \int_{-\infty}^{\infty} (e^x - e^k)e^{iuk}e^{\alpha k}dk = e^x \int_{-\infty}^{x}e^{(\alpha +iu)k}dk -\int_{-\infty}^{x}e^({\alpha + iu + 1)k}dk, \]
	\[ = \frac{e^x}{\alpha + iu}[ e^{(\alpha + iu)k} ]^x_{-\infty} - \frac{1}{\alpha + iu + 1}[e^{(\alpha + iu + 1)k}]^x_{-\infty} \]
	En prenant la limite pour $lim_{k \rightarrow -\infty} e^{(\alpha + iu)k} = 0$ avec $\alpha > 0$ on a :
	\[  \left(   \frac{e^{(\alpha+1+iu}x}{\alpha + iu} - \frac{e^{(\alpha+1+iu}x}{\alpha + iu + 1}        \right)  \]
	Ce qui nous amène :
	\[ \Psi(u) = \int_{-\infty}^{\infty}e^{-rT}q(x)  \left(   \frac{e^{(\alpha+1+iu}x}{\alpha + iu} - \frac{e^{(\alpha+1+iu}x}{\alpha + iu + 1}        \right)dx, \]
	\[ = e^{-rT}\int_{-\infty}^{\infty} q(x) \frac{e^{(\alpha + 1 + iu)x}}{(\alpha+iu)(\alpha+1+iu)}dx. \]
	En prenant la transformé de Fourier pour : $\int_{-\infty}^{\infty}q(x)e^{(\alpha + 1 +iu)x} = \int_{-\infty}^{\infty}q(x)e^{i(u-(\alpha+1))x}$ on a la fonction caractéristique pour le prix sous la probabilité risque neutre $\phi_T(u-(\alpha+1)i)$ finalement on a :
	\[ \Psi(u) = \frac{e^{-rT} \phi_T(u-(\alpha+1)i)  }{ \alpha^2 + \alpha -u^2 + i(2\alpha + 1)u^\prime  } \]
	On a donc : 
	\[ C_T(k) = \frac{e^{-\alpha k}}{2\pi}\int_{-\infty}^{\infty} e^{-iuk}\Psi(u)du, \]
	\[ = \frac{e^{-\alpha k}}{\pi}\int_{0}^{\infty}\Re[e^{-iuk} \Psi(u)]du \]
	
	En connaissant cela, on peut répliquer de manière théorique, toutes fonctions en utilisant uniquement des Calls et des Puts (Continuum Call/Put).
	
	

   
   \section{Code Python pour simulations et estimations}
   
   
   \includepdf[pages=1, pagecommand={\subsection{Code Python pour l'estimation du paramètre de Hurst H}}, scale=0.8, trim=0 2cm 0 2cm, clip]{EstimationParamH.pdf}
   \includepdf[pages=2-, pagecommand={}, scale=0.8, trim=0 2cm 0 2cm, clip]{EstimationParamH.pdf}   
  
   \includepdf[pages=1, pagecommand={\subsection{Code Python pour les simulations du mouvement brownien fractionnaire et du processus d'Ornstein-Uhlenbeck}}, scale=0.8, trim=0 2cm 0 2cm, clip]{MBFetPOU.pdf}
   \includepdf[pages=2-, pagecommand={}, scale=0.8, trim=0 2cm 0 2cm, clip]{MBFetPOU.pdf}
   
   \includepdf[pages=1, pagecommand={\subsection{Code Python pour les simulations du Modèle de Black-Scholes fractionnaire et du modèle Cox-Ingersoll-Ross}}, scale=0.8, trim=0 2cm 0 2cm, clip] {CIRetBS.pdf}
   \includepdf[pages=2-, pagecommand={},  scale=0.8, trim=0 2cm 0 2cm, clip] {CIRetBS.pdf}
   
  
   \section{Bibliographie}
   \begin{thebibliography}{99}

	\bibitem{FN}
	{Mathieu Rosenbaum, Jim Gatheral, Thibault Jaisson} {\textbf{\em Volatility is rough }}
	
	\bibitem{F}
	{Masaaki Fukasawa, Jim Gatheral} {\textbf{\em A rough SABR formula }}
	
	\bibitem{C}
	{Mathieu Rosenbaum, Omar El Euch} {\textbf{\em The characteristic function of rough Heston Models}}
	
	\bibitem{Dab10}
	{Elisa Alos, David Garcia-Lorite, Aitor Muguruza} {\textbf{\em On smile properties of volatility derivatives and exotic products: understanding the VIX skew}}
	
	\bibitem{FCM}
	{Mathieu Rosenbaum, Paul Jusselin} {\textbf{\em No-arbitrage implies power-law market impact and rough volatility}}

	\bibitem{V}
	{Fabienne Comte, Eric Renault} {\textbf{\em LONG MEMORY IN CONTINUOUS-TIME STOCHASTIC VOLATILITY MODELS }}
	
	\bibitem{F}
	{Gaël Didier} {\textbf{\em Application du mouvement Brownien fractionnaire à l’évaluation des engagements sur des contrats en unité de compte }}

	\bibitem{C}
	{Ton Dieker} {\textbf{\em Simulation of Fractional Brownian Motion}}

	\bibitem{C}
	{Jean-François Coeurjolly} {\textbf{\em Inférence statistique pour les mouvements browniens fractionnaires et multifractonnaires}}	
		
	\bibitem{C}
	{Christian Bender} {\textbf{\em An Itô formula for generalized functionals of a fractional Brownian motion with arbitrary Hurst parameter}}	

	\bibitem{C}
	{Mathieu Rosenbaum} {\textbf{\em Rough Heston models : Pricing and hedging}}	
	
	\bibitem{C}
	{Peter Carr, Dilip B.Madan} {\textbf{\em Option Valuation using the fast Fourier Transform}}	
	
	\bibitem{C}
	{Omar El Euch} {\textbf{\em Quantitative Finance Under Rough Volatility}}	
	
	\bibitem{C}
	{David Nualart} {\textbf{\em The Malliavin Calculus And Related Topics}}	
	
	\bibitem{C}
	{Mathieu Rosenbaum, Omar El Euch} {\textbf{\em Perfect hedging in rough Heston models}}	
	
	\bibitem{C}
	{Elisa Alos, Kenichiro Shiraya} {\textbf{\em Estimating the Hurst parameter from short term volatility swaps: a Malliavin calculus approach }}	
	
	\bibitem{C}
	{Fabrice Douglas Rouah} {\textbf{\em The SABR model}}	
	
	\bibitem{C}
	{Gazanfer Unal, Ali Dinler} {\textbf{\em Exact linearization of one dimensional Itô equations driven by fBm: Analytical and numerical solutions}}	
	
	\bibitem{C}
	{Sami El Rahouli } {\textbf{\em Financial modeling with Volterra processes and applications to options, interest rates and credit risk }}	
	
	\bibitem{C}
	{Marc M. Mpanda, Safari Mukeru, Mmboniseni Mulaudzi} {\textbf{\em Generalisation of Fractional Cox-Ingersoll-Ross Process }}	
	
	\bibitem{C}
	{Yuliya Mishura, Anton Yurchenko-Tytarenko} {\textbf{\em Fractional Cox–Ingersoll–Ross process with non-zero «mean» }}	
	
	\bibitem{C}
	{David Nualart, Samy Tindel} {\textbf{\em A CONSTRUCTION OF THE ROUGH PATH ABOVE FRACTIONAL BROWNIAN MOTION USING VOLTERRA’S REPRESENTATION}}	
	
	\bibitem{C}
	{Christian Bayer, Peter Friz, Jim Gatheral} {\textbf{\em Pricing under rough volatility}}	
	
	\bibitem{C}
	{Mathieu Garcin} {\textbf{\em Hurst exponents and delampertized fractional Brownian motions}}	
	
	\bibitem{C}
	{Eyal Neuman, Mathieu Rosenbaum} {\textbf{\em Fractional Brownian motion with zero Hurst parameter: a rough volatility viewpoint }}	


   


\end{document}
