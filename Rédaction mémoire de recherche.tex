\documentclass[french]{article}
\usepackage[utf8]{inputenc}
\usepackage[T1]{fontenc}
\usepackage{babel}
\usepackage{amssymb}
\usepackage{amsmath}
\usepackage{titlesec}

\date{September 23, 2021}
\author{\bsc{Réda Bensahla - Université de Lille} }

\title{Le Mouvement Brownien Fractionnaire appliqué à la Log-Volatilité des actifs financiers (Rough Volatility)}

\begin{document}
   \maketitle 
   \newpage
   \renewcommand{\contentsname}{Sommaire}
      \tableofcontents

   \newpage   
   \section{Introduction}

      \subsection{L'utilisation du mouvement brownien dans les modèles financiers modernes}
	Dans la modélisation financière moderne, les prix ont toujours été modélisés par des martingales semi-continues, de plus, il a toujours été nécessaire de présenter certaines hypothèses, qui aujourd'hui peuvent être remises en cause, notamment :
	\begin{itemize}
    		\item L'hypothèse de normalité des rendements, ceux-ci présentent en général des asymétries et des queues épaisses.
   		\item La continuité des cours, pour cela il suffit d'observer les sauts, plus communément appelés "gaps" entre deux instants de quotations de cours.
    		\item Mais surtout, et le plus important, l'hypothèse d'indépendance des accroissements, en effet, il s'agit ici d'ignorer l'impacts des évènements réalisés dans le passé, pour cela, Mandelbrot préconise d'utiliser le mouvement Brownien Fractionnaire à la place du mouvement Brownien.
	\end{itemize} 
	C'est donc ce point qui nous a amené à faire cette étude sur le mouvement Brownien Fractionnaire, de plus, la Log-Volatilité étant fractionnaire l'utilisation de ce mouvement Brownien nous donne donc des modèles assez consistant, cependant, comme nous le verrons par la suite, il est assez difficile à mettre en place.
     
      \subsection{Modélisation de la volatilité}
      Les Logs-prix sont souvent modélisés comme des semi-martingales continues. Pour un actif donné avec un Log-Prix $Y_t$, ce dernier peut être modélisé comme suit : \[ dY_t = \mu_t dt + \sigma_t dWt\] où $\mu_t$ est le terme de drift et $W_t$ est un mouvement brownien uni-dimensionnel.
      Le terme $\sigma_t$ représente quant à lui le processus de volatilité du modèle.
      D'un côté nous avons des modèles comme le modèle de Black-Scholes où la volatilité est souvent constante ou est une fonction déterministe du temps, puis d'un autre côté, nous avons des modèles à volatilité stochastiques, la volatilité $\sigma_t$ est modélisée par une semi-martingale brownienne, parmi ces modèles nous pouvons citer le                                 	modèle de Heston dont voici la dynamique de la volatilité: \[ d\nu_t = \kappa(\theta - \nu_t)dt + \xi \sqrt{\nu_t} dB_t\] Ou encore le modèle CEV ("Constant Elasticity of Variance Model") dont la volatilité est stochastique mais déterministe, sa dynamique s'écrit comme suit :  \[ dS_t = \mu S_tdt + \sigma S_t^\gamma dB_t\]
      
      \subsection{Le mouvement brownien fractionnaire}
      	\subsubsection{Définition}
     	  Dans cette section, nous allons définir le mouvement brownien fractionnaire, ses propriétés, comment le simuler et enfin comment estimer l'index de Hurst (H).\newline \newline Le mouvement brownien fractionnaire noté $\{B_H(t)\}_{t\in\mathbb{R}}$ d'exposant de Hurst $H\in ]0;1]$, est l'unique processus gaussien centré, nul en zéro, dont les 	  accroissements sont stationnaires et auto-similaires, il est défini par  :  \[ B_H(t) := \int_{\mathbb{R}} [ (t-s)_+^{H-\frac{1}{2}}-(-s)_+^{H-\frac{1}{2}}\ ]\mathrm{d}B(s),\] 
	  Où : $B_H(0) = 0$ et $B(s)$ représente le mouvement brownien standard
      	
	\subsubsection{Fonction de covariance et d'auto-covariance}
     	  Sa fonction de covariance $\Gamma(t,s)$ et d'autocovariance $\gamma(t,s)$ sont données par :
     	   \[  \Gamma(t,s) = \mathbb{E} [B_H(s)B_H(t)] = \frac{C_H}{2}(\lvert t \rvert^{2H}  +   \lvert s \rvert^{2H}      +     \lvert t - s \rvert^{2H}))  \]
	   \[  \gamma(t,s) =  \frac{C_H}{2}(\lvert t-s-1 \rvert^{2H}  -   2\lvert t-s \rvert^{2H}      +     \lvert t-s+1 \rvert^{2H}))  \]
      	  où \[ C_H = Var[B_H(1)]\] 
    	  Lorsque $C_H = 1$ ce processus est appelé Mouvement Brownien Fractionnaire standard, de plus lorsque H = $\frac{1}{2}$, le mouvement brownien fractionnaire correspond à un mouvement brownien standard \newline
	  Le comportement du mouvement brownien fractionnaire dépend donc de ce paramètre H : 
	  \begin{itemize}
    		\item Lorsque $H < \frac{1}{2}$ les trajectoires sont assez irrégulières (comparativement au mouvement brownien standard).
   		\item Lorsque $H > \frac{1}{2}$ les trajectoires sont plus régulières, on constate l'effet de longue mémoire.
	\end{itemize} 
	  
	  \subsubsection{Simulation du mouvement brownien fractionnaire} 
	  Nous allons donc présenter ici une méthode de simulation du mouvement brownien fractionnaire : La méthode de Cholesky \newline
	  	\paragraph{Méthode de Cholesky}
		Il s'agit ici d'une méthode qui est exacte en théorie, mais qui n'est malheureusement pas très utile en pratique car son execution nécessite beaucoup de temps. \newline
		Posons $\Gamma$ la matrice de covariance du mouvement brownien discrétisé aux instants $\frac{i}{n}$, pour $i = 0,\ldots,n-1.$ \newline
		La méthode de Cholesky consiste à déduire $\Gamma^\prime$ de $\Gamma$ en supprimant la première ligne et la première colonne, la matrice $\Gamma^\prime$ est symétrique définie positive et admet donc
		une décomposition de Cholesky $\Gamma^\prime = LL^t$ où $L$ est est une matrice triangulaire inférieure.\newline \newline
		Par la suite, on effectue le produit matriciel $LZ$ où $Z$ est un vecteur de $n-1$ variables aléatoires indépendantes gaussiennes centrées et réduite, le vecteur $LZ$ est un vecteur gaussien centré et on a : $\mathbb{E}[(LZ)(LZ)^t] = \Gamma^\prime$ \newline \newline
	  	Ainsi, le vecteur $B_H = (0, (LZ)^t)^t$ représente une trajectoire du mouvement brownien discrétisé.
		
	 \subsubsection{Le processus d'Ornstein-Uhlenbeck associé au Mouvement Brownien Fractionnaire}
     	   Le processus de Ornstein-Ulhenbeck associé au mouvement brownien fractionnaire associé est donné par l'équation suivante : 
      	   \[ dx = \theta(t)(\mu (t)(t)-x)dt + \sigma (t) dB^h_t\] 
	   
	   

			
	   
      \subsubsection{Simulations du mouvement brownien fractionnaire et du processus d'Ornstein-Uhlenbeck associé}
      	  Ce processus est très utile pour modéliser des taux, voici une simulation de ce processus pour : 
     	 (Simulation)
      
      \subsection{La volatilité rugueuse}
      	\subsubsection{Définition}
      		Les différents modèles historiques dans la modélisation financière moderne proposent deux approches concernant le processus de volatilité $\sigma_t$ : 
		\begin{itemize}
    			\item Premièrement, des trajectoires assez régulières (constantes ou déterministes), dans le cas du modèle de Black-Scholes standard.
   			\item Deuxièmement, dans le cas des modèles à volatilité locales ou stochastiques, comme le modèle de Heston, les trajectoires de volatilité modélisées correspondaient pratiquement à celles du mouvement brownien standard.
		\end{itemize}
	Dans l'article de Comte et Renault, les auteurs proposent, en partant du fait que la volatilité est un processus à mémoire-longue, de modéliser la log-volatilité en utilisant un mouvement brownien fractionnaire avec pour paramètre $H \in ]\frac{1}{2},1[$, ce modèle est appelé Fractional Stochastic Volatility (FSV) \newline \newline
	
	Dans l'article de Gatheral, Jaisson et Rosenbaum, un modèle plus complexe est proposé, le Rough Fractional Stochastic Volatility (RFSV), celui-ci est le même que le modèle FSV mais en prenant $H \in ]0; \frac{1}{2}[$, nous détaillerons ces modèles dans la section ().\newline \newline
	
	Nous utiliserons par la suite le mouvement brownien fractionnaire afin de modéliser les incréments de la log-volatilité, en partant du principe que les incréments stationnaires du mouvement brownien fractionnaire satisfont, pour tout $t \in \mathbb{R}, \Delta \geq 0, q>0$ : 
	 \[ \mathbb{E}[\lvert B_{t+\Delta}^H -  B_t^H \rvert^q]  = K_q\Delta^{qH}     \] 
	 Lorsque $H > \frac{1}{2}$, les incréments du mouvement brownien fractionnaire sont positivement corrélés et montre un effet de mémoire longue, en effet, la trajectoire de $B_{t+1}^H$ dépend de $B_t^H$, et on a donc :  	 \[  \sum_{k=0}^{+\infty}Cov[W_1^H, W_k^H - W_{k-1}^H] = +\infty   \]  \newline
	 En effet, $Cov[W_1^H, W_k^H - W_{k-1}^H]$ est de l'ordre $k^{2H-2}$ avec $k\longrightarrow \infty$
	 
      	Où $K_q$ représente le moment d'ordre $q$ de la valeur absolue d'une variable gaussienne centrée et réduite; $\Delta$ représente la variation temporelle. \newline \newline
	L'approche de Gatheral et Rosenbaum est de partir du principe que l'on a accès à différentes observations du processus de volatilité $\sigma_t$ et de découper les instants de volatilités sur $[0;T]$ comme suit : $\sigma_0, \ldots, \sigma_\Delta, \dots, \sigma_{k\Delta}, \dots: k \in {[0,  N]}$, où $N = \left\lfloor {T/\Delta} \right\rfloor$.
	
	On définit donc, pour $q \geq 0$ :
	\[ m(q, \Delta) = \frac{1}{N} \sum_{k=1}^N\lvert log(\sigma_{k\Delta}) - log(\sigma_{(k-1)\Delta}) \rvert^{q}  \] 
	
	\subsubsection{Estimation de l'index de Hurst H}
	(Graphiques à mettre pour appuyer les résultats) \newline
	La méthode d'estimation du paramètre H proposée par Gatheral, Jaisson et Rosenbaum se construit comme suit : \newline \newline
	Il faut comprendre le comportement de la quantité $\frac{log \; m(q,\Delta)}{log(\Delta)}$, on s'aperçoit que cette relation est linéaire, à un coefficient $\zeta_q$ près, on conclut que pour tout $q : \; m(q, \Delta) \propto \Delta^{\zeta_q} $ \newline
	Enfin il s'agit d'étudier le comportement de $\zeta_q$ en fonction de $q$, nos résultats nous montrent qu'il existe donc une relation linéaire entre $\zeta_q$ et $q$, de ce résultat est déduit la pente la droite qui correspond donc au paramètre H, et on déduit donc la relation d'échelle monofractal suivante :  \[  \zeta_q = qH \]
	Il convient de remarquer que H varie très peu en fonction du temps.

	\subsubsection{Application sur un actif quelconque}
	
   \subsection{Équations Différentielles Stochastiques dirigées par un mBF}
   Dans cette section nous allons nous intéresser à la résolution théorique de certaines EDS ou lorsque la solution théorique n'est pas atteignable, nous présenterons un schéma de résolution numérique (Schéma de Milstein).
   Enfin, nous parlerons des équations dirigés par le mouvement brownien fractionnaire tels que le processus d'Ornstein-Uhlenbeck, le modèle CIR ou encore le modèle de Black Scholes fractionnaire.
   	\subsubsection{Présentation sous la forme différentielle et sous la forme intégrale}
		L'équation différentielle stochastique dirigés par un mouvement brownien fractionnaire est donnée par :  \[  dY_t = f(Y_t, t)dt + g(Y_t, t)dB_t^H  \] 
		Où $B_t^H$ est un mouvement brownien fractionnaire standard. \newline 
		Une présentation alternative, sous forme d'intégrale :
		  \[ Y_t = Y_s + \int_{s}^{t}f(Y_u, u)du + \int_{s}^{t}g(Y_u, u)dB_u^H \tag{1} \] 
		Pour le cas où $H=\frac{1}{2}$ l'intégrale stochastique de l'équation (1) est une martingale, ainsi le le processus $Y_t$ est, au minimum, une semi-martingale, ainsi la formule d'Itô est applicable. \newline
		Cependant, dans le cas où $H \ne \frac{1}{2}$ il est nécessaire de discuter les cas en fonction de H : 
		\begin{itemize}
    			\item Dans le cas où $H > \frac{1}{2}$ 
   			\item Dans le cas où $H \in ]\frac{1}{4};\frac{1}{2}]$
		\end{itemize}
	\subsubsection{Chemins rugueux}
	
	\subsubsection{Schéma numérique de Milstein}
		Une EDS ne possède pas souvent une solution théorique explicite (formule fermée), et il est donc parfois nécessaire d'avoir recours à des Schémas numérique de calcul, dans le cas où nous avons une EDS dirigée par un mouvement brownien standard, il peut être judicieux d'utiliser un Schéma numérique d'Euler. \newline
		Le Schéma de Milstein est valable pour tout $H \in ]0;1[$
		
	
   \newpage

   
   \section*{Conclusion}
   \addcontentsline{toc}{section}{Conclusion}
   
   \appendix 
   \section{Preuves}

  
   \section{Bibliographie}



\end{document}
