\documentclass[french]{article}
\usepackage[utf8]{inputenc}
\usepackage[T1]{fontenc}
\usepackage{babel}
\usepackage{amssymb}
\usepackage{amsmath}
\usepackage{titlesec}

\date{September 23, 2021}
\author{\bsc{Réda Bensahla - Université de Lille} }

\title{Le Mouvement Brownien Fractionnaire appliqué à la Log-Volatilité des actifs financiers (Rough Volatility)}

\begin{document}
   \maketitle 
   \newpage
   \renewcommand{\contentsname}{Sommaire}
      \tableofcontents

   \newpage   
   \section{Introduction}

      \subsection{L'utilisation du mouvement brownien dans les modèles financiers modernes}
	Dans la modélisation financière moderne, les prix ont toujours été modélisés par des martingales semi-continues, de plus, il a toujours été nécessaire de présenter certaines hypothèses, qui aujourd'hui peuvent être remises en cause, notamment :
	\begin{itemize}
    		\item L'hypothèse de normalité des rendements, ceux-ci présentent en général des asymétries et des queues épaisses.
   		\item La continuité des cours, pour cela il suffit d'observer les sauts, plus communément appelés "gaps" entre deux instants de quotations de cours.
    		\item Mais surtout, et le plus important, l'hypothèse d'indépendance des accroissements, en effet, il s'agit ici d'ignorer l'impacts des évènements réalisés dans le passé, pour cela, Mandelbrot préconise d'utiliser le mouvement Brownien Fractionnaire à la place du mouvement Brownien.
	\end{itemize} 
	C'est donc ce point qui nous a amené à faire cette étude sur le mouvement Brownien Fractionnaire, de plus, la Log-Volatilité étant fractionnaire l'utilisation de ce mouvement Brownien nous donne donc des modèles assez consistant, cependant, comme nous le verrons par la suite, il est assez difficile à mettre en place.
     
      \subsection{Modélisation de la volatilité}
      Les Logs-prix sont souvent modélisés comme des semi-martingales continues. Pour un actif donné avec un Log-Prix $Y_t$, ce dernier peut être modélisé comme suit : \[ dY_t = \mu_t dt + \sigma_t dWt\] où $\mu_t$ est le terme de drift et $W_t$ est un mouvement brownien uni-dimensionnel.
      Le terme $\sigma_t$ représente quant à lui le processus de volatilité du modèle.
      D'un côté nous avons des modèles comme le modèle de Black-Scholes où la volatilité est souvent constante ou est une fonction déterministe du temps, puis d'un autre côté, nous avons des modèles à volatilité stochastiques, la volatilité $\sigma_t$ est modélisée par une semi-martingale brownienne, parmi ces modèles nous pouvons citer le                                 	modèle de Heston dont voici la dynamique de la volatilité: \[ d\nu_t = \kappa(\theta - \nu_t)dt + \xi \sqrt{\nu_t} dB_t\] Ou encore le modèle CEV ("Constant Elasticity of Variance Model") dont la volatilité est stochastique mais déterministe, sa dynamique s'écrit comme suit :  \[ dS_t = \mu S_tdt + \sigma S_t^\gamma dB_t\]
      
      \subsection{Le mouvement brownien fractionnaire}
      	\subsubsection{Définition}
     	  Dans cette section, nous allons définir le mouvement brownien fractionnaire, ses propriétés, comment le simuler et enfin comment estimer l'index de Hurst (H).\newline \newline Le mouvement brownien fractionnaire noté $\{B_H(t)\}_{t\in\mathbb{R}}$ d'exposant de Hurst $H\in ]0;1]$, est l'unique processus gaussien centré, nul en zéro, dont les 	  accroissements sont stationnaires et auto-similaires, il est défini par  :  \[ B_H(t) := \int_{\mathbb{R}} [ (t-s)_+^{H-\frac{1}{2}}-(-s)_+^{H-\frac{1}{2}}\ ]\mathrm{d}B(s),\] 
      	
	\subsubsection{Fonction de covariance et d'auto-covariance}
     	  Sa fonction de covariance $\Gamma(t,s)$ et d'autocovariance $\gamma(t,s)$ sont données par :
     	   \[  \Gamma(t,s) = \mathbb{E} [B_H(s)B_H(t)] = \frac{C_H}{2}(\lvert t \rvert^{2H}  +   \lvert s \rvert^{2H}      +     \lvert t - s \rvert^{2H}))  \]
	   \[  \gamma(t,s) =  \frac{C_H}{2}(\lvert t-s-1 \rvert^{2H}  -   2\lvert t-s \rvert^{2H}      +     \lvert t-s+1 \rvert^{2H}))  \]
      	  où \[ C_H = Var[B_H(1)]\] 
    	  Lorsque $C_H = 1$ ce processus est appelé Mouvement Brownien Fractionnaire standard, de plus lorsque H = $\frac{1}{2}$, le mouvement brownien fractionnaire correspond à un mouvement brownien standard \newline \newline
	  
	  De plus, lorsque H = $\frac{1}{2}$, $\forall \lvert k \rvert \geq 1$ $\gamma(k) = 0$, ainsi les accroissements sont donc indépendants
	  
	  \subsubsection{Simulation du mouvement brownien fractionnaire} 
	  Nous allons donc présenter ici quatre méthodes de simulations du mouvement brownien fractionnaire : (Cholesky, et Davies and Harte) \newline
	  
	  
  	 \subsubsection{Le processus d'Ornstein-Uhlenbeck associé au Mouvement Brownien Fractionnaire}
     	   Le processus de Ornstein-Ulhenbeck associé au mouvement brownien fractionnaire associé est donné par l'équation suivante : 
      	   \[ dx = \theta(t)(\mu (t)(t)-x)dt + \sigma (t) dB^h_t\] 
	   
	   
	 \subsubsection{Estimation de l'index de Hurst H} 
	 On suppose les éléments suivants connus : 
	 \begin{itemize}
    		\item La trajectoire d'un mouvement brownien fractionnaire $B_H$ de taille n
   		\item Le vecteur des accroissements, que l'on appellera X
	\end{itemize}
	Nous allons présenter trois méthodes d'estimation de H, commençons par la méthode du maximum de vraisemblance.
		\paragraph{Estimation par maximum de vraisemblance : Estimateur de Whitle}
			
	   
      \subsubsection{Simulations du mouvement brownien fractionnaire et du processus d'Ornstein-Uhlenbeck associé}
      	  Ce processus est très utile pour modéliser des taux, voici une simulation de ce processus pour : 
     	 (Simulation)
      
      \subsection{La volatilité rugueuse}
         
   \newpage
   \section{Une deuxième section avec une sous-section}
      texte d’introduction de la deuxième section
      \subsection{Sous-section 1}
         texte de la sous-section 1
      \subsection{Sous-section 2}
         texte de la sous-section 2
   
   \section*{Conclusion}
   \addcontentsline{toc}{section}{Conclusion}
      texte de la conclusion 
   
   \appendix 
   \section{Preuves}
       texte de l’annexe 1 
  
   \section{Bibliographie}
      texte de l’annexe 2


\end{document}
